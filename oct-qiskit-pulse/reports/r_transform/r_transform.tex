\documentclass[12pt]{article}
\usepackage{physics}

\title{Rotating Wave Approximation for QOC}
\author{
        Ben Rosand \\
        IBM Quantum \\
        Quantum Intern -- Pulse Team\\
        Yorktown Heights, NY
}
\date{\today}


\begin{document}
\maketitle


\begin{align}
        & H_{lab} = H_0 + \sum_m^M \Omega_m \cos{(\omega_m t + \phi_m)} \sum_S g \sigma^x\\
        \begin{split}
            H_{lab} = H_0 + \big[\Omega_{D0} \cos{(\omega_{D0} t + \phi_{D0})}\\
                          + \Omega_{D1} \cos{(\omega_{D1} t + \phi_{D1})}\\
                          + \big(\Omega_{U0} \cos{(\omega_{U0} t + \phi_{U0})}\\
                          + \Omega_{U1} \cos{(\omega_{U1} t + \phi_{U1})}\big)\big]\\
                          * \sigma^x
        \end{split} \\
        % & H_{lab} = H_0 +
        \begin{split}
            H_{rot} = H_0 + \frac{dR}{dt} \big[\Omega_{D0} + e^{-iR} \cos{(\omega_{D0} t + \phi_{D0})}\\
                          + \Omega_{D1} \cos{(\omega_{D1} t + \phi_{D1})}\\
                          + \Omega_{U0} \cos{(\omega_{U0} t + \phi_{U0})}\\
                          + \Omega_{U1} \cos{(\omega_{U1} t + \phi_{U1})}\big]\\
                          * \sigma^x e^{iR}
        \end{split} \\
        & H_{rot} = H_0 + \frac{dR}{dt}+ \frac{1}{2}
         \sum_{m=1}^2 \Omega_m \qty{\cos{(\phi_m)} \sigma^x - \sin{(\phi_m)} \sigma^y}\\
\end{align}


For the transformation we require that:
\begin{equation}\label{eq:6}
    \begin{split}
        e^{-iR} (\cos{(\omega_m t + \phi_m)}) \sigma^x - \sin{(\omega_m t + \phi_m)} \sigma^y) e^{iR} \\
        = \cos{\phi_m} \sigma^x - \sin{\phi_m} \sigma^y
    \end{split}
\end{equation}
With $m = {D0, D1}$, we only rotate on these two axis\\
Then when we perform the transformation it is just applied separately to each drive and control

Following from that we have the requirement
\begin{equation}
    \begin{split}
        \cos(\theta) \sigma^x + \sin(\theta) \sigma^y\\
        = \cos(\omega_m t) \sigma^x + \sin(\omega_m t) \sigma^y
    \end{split}
\end{equation}


We have four fields, D0, D1, U0, U1
Note we don't transform the U control fields
\subsection{Finding R}
\begin{align}
   &[R, \sigma^x] = \omega t (i \sigma^y) \\
\end{align}
\begin{equation}
    \begin{bmatrix}
        1 \\
        1
    \end{bmatrix}
    \begin{bmatrix}
        c_{D0} \\
        c_{D1}
    \end{bmatrix}
     =
    \begin{bmatrix}
        \omega_1 t \\
        \omega_2 t
    \end{bmatrix}
\end{equation}

\begin{align}
    c_{D0} + c_{D1} = \omega_1 t\\
    c_{D0} + c_{D1} = \omega_2 t
\end{align}

This shouldn't be right though, because it implies that $\omega_1 = \omega_2$\\

We have states $\ket{00}$ through $\ket{22}$, and can transition from 01, 02, 03
to 11, 12, 13, (we are kind of ignoring the 1-2 transition, because we are using
the general sigmax and hoping it counts for both, but)NOTE: This might be a
place to come back and switch to a linear combination if it isn't working

what if we want to do linear combination? The worry is that there aren't
different drives, it's the same 

% So does this mean R is just $\sigma^z (\omega_1 + \omega_2) t$
Let's test this. First does it fit into \eqref{eq:6}?

\subsection{Figuring out R from pdf}

Use 3.13 for braket definition of pauli matrices
\begin{align}
    & R = \sum c_s \sigma^z_s \label{eq:R_expansion}\\
    & [R, \sigma^x_{nn'}] = \sum c_{mm'} [\sigma^z_{mm'}, \sigma^x_{nn'}]\\
    & [\sigma^z_{mm'}, \sigma^x_{nn'}] = [(\ket{m} \bra{m} - \ket{m'} \bra{m'}), (\ket{n} \bra{n'} + \ket{n'}\bra{n})]
\end{align}

Now we notice that there are a number of cases, depending on whether m, m', n,
and n' are equal to or different from each other. Fischer lays out the
evaulation for each of these options.

The commutator simplifies to

\begin{equation}\label{eq:commutator_cases}
    [\sigma^z_{mm'}, \sigma^x_{nn'}] = 
    \begin{cases}
        +2 i \sigma^y_{nn'} & m=n, m'=n' \\
        + i \sigma^y_{nn'} & m=n, m' \neq ',\\
        & \text{or } m \neq n, m'=n'\\
        - i \sigma^y_{nn'} & m=n' or m'=n \\
        0 & m \neq n, m' \neq n'
    \end{cases}
\end{equation}

Given this, we can note that $[R, \sigma^x_{nn'}]$ is always proportional to $i
\sigma^y_{nn'}$

\begin{equation}\label{eq:theta_commute}
    [R, \sigma^x_{nn'} = \theta_{nn'} i\sigma^y_{nn'}]
\end{equation}
For some constant $\theta_{nn'}$, and we look at eq (3.19) from Fischer to evaluate 3.31

\begin{equation} \label{eq:Fischer_3.38}
    e^{-i R} \sigma^x_{nn'} e^{iR} = \cos{(\theta_{nn'})}\sigma^x_{nn'} + \sin{(\theta_{nn'})} \sigma^y_{nn'}
\end{equation}
note that the i from eq 3.19 cancels out with the i from \eqref{eq:theta_commute}

Now we equate the right hand side of \eqref{eq:Fischer_3.38} with the right hand side of (F:3.31) to get
\begin{equation}\label{eq:theta_eq_wt}
    \cos{(\theta_{nn'})}\sigma^x_{nn'} + \sin{(\theta_{nn'})} \sigma^y_{nn'}
    = \cos{(\omega t)}\sigma^x_{nn'} + \sin{(\omega t)} \sigma^y_{nn'}
\end{equation}


Our goal is to satisfy equation 3.29 from Fischer, which is the one in which
$e^{-iR} \{cos - sin\} e^{iR} = cos - sin$

So thus in order to satisfy equation 3.29 we must satisfy
\eqref{eq:theta_eq_wt}. 
This only holds if $\theta_{nn'} = \omega t$ for all
$(n,n') S$

\begin{equation}
    [R, \sigma^x_{nn'}] = \omega t (i \sigma^y_{nn'})
\end{equation}


We use the expansion \eqref{eq:R_expansion} for R, and the commutations from
\eqref{eq:commutator_cases}, we obtain a matrix equation to solve for the
coefficients.

\begin{align}
    & R = \sum c_s \sigma^z_s \label{eq:R_expansion}\\
    & [\sigma^z_{mm'}, \sigma^x_{nn'}] = 
    \begin{cases}
        +2 i \sigma^y_{nn'} & m=n, m'=n' \\
        + i \sigma^y_{nn'} & m=n, m' \neq ',\\
        & \text{or } m \neq n, m'=n'\\
        - i \sigma^y_{nn'} & m=n' or m'=n \\
        0 & m \neq n, m' \neq n'
    \end{cases}\\
    & [R, \sigma^x_{nn'}] = \omega t (i \sigma^y_{nn'})
\end{align}

Thus:
\begin{equation}
    \qty{[c_{mm'} \sigma^z_{mm'}, \sigma^x_{nn'}] = i \omega t \sigma^y_{nn'}} \forall (m,m') in S \forall (n,n') \in S
\end{equation}

And we can solve using \eqref{eq:commutator_cases}

So let's take the case where we have 2 drive channels, and either general $\sigma^x$ or $\sigma^x_{01} + \sigma^x_{12}$

For both options, we have 9 states, and since we have two separate control
 fields which effect orthogonal state transitions, the derivation can proceed at
 least temporarily without taking into account the multiple control fields.

Let's follow the single field derivation first, because that is the first step either way.

for the $\ket{00}\rightarrow \ket{01}$ transition we obtain the following equation:

\textbf{potential error}: we ignore skipping between 0 and 2 (00) and (02), because not really allowed
\newpage
quick reference
\begin{itemize}
    \item 00 -- 0
    \item 01 -- 1
    \item 02 -- 2
    \item 10 -- 3
    \item 11 -- 4
    \item 12 -- 5
    \item 20 -- 6
    \item 21 -- 7
    \item 22 -- 8
\end{itemize}
\textbf{POtential error}: We ignore reversals, cause always reversible? Seems
like they are just negative of each other?

\textbf{NOTE}: we move from the two index basis to single, so $\ket{00}$ becomes
$\ket{0}$ and $\ket{01}$ becomes $\ket{1}$ thus we can easily
show $\ket{0} \rightarrow \ket{1}$ as $01$
\begin{align}
    & c_{00,01}(2 i \sigma^y_{00,01})
    + c_{00,10}(i \sigma^y_{00,01}) 
    + c_{01,02}(- i \sigma^y_{00,01}) 
    + c_{01,11}(- i \sigma^y_{00,01}) 
    % + c_{02,01}(i \sigma^y_{00,01}) \text{ if we have reversals}
    % + c_{11,01}(i \sigma^y_{00,01})\text{ if we have reversals}
    = \omega t (i \sigma^y_{00,01})\\
    & c_{00,10}(2 i \sigma^y_{00,10})
    + c_{00,01}(i \sigma^y_{00,10}) 
    + c_{10,20}(- i \sigma^y_{00,10}) 
    + c_{10,11}(- i \sigma^y_{00,10}) 
    % + c_{02,01}(i \sigma^y_{00,01}) \text{ if we have reversals}
    % + c_{11,01}(i \sigma^y_{00,01})\text{ if we have reversals}
    = \omega t (i \sigma^y_{00,01})
    & \omega t
\end{align}

Now convert to 

\begin{equation}
      c_{01}(2 i \sigma^y_{01})
    + c_{03}(i \sigma^y_{01})
    + c_{12}(- i \sigma^y_{01}) 
    + c_{14}(- i \sigma^y_{01})
    % + c_{02,01}(i \sigma^y_{00,01}) \text{ if we have reversals}
    % + c_{11,01}(i \sigma^y_{00,01})\text{ if we have reversals}
    = \omega t (i \sigma^y_{01})
\end{equation}

% The class \texttt{rwa\_system_solver} is \ldots
We do the math in python, see \texttt{rwa\_system\_solver.py}

So now that we have the system of equations we need to solve
so with $M$ being the matrix of values from the commutations
\begin{equation}
    M[c_{nn'}] = [\omega_i t]
    \begin{cases}
        i = 1 & \text{for transitions on qubit 1}\\
        i = 2 & \text{for transitions on qubit 2}
    \end{cases}
\end{equation}

\begin{equation}
    \begin{split}
    H(t) = H_0 + \Omega_{d0} D_0(t) \sigma_0^x + \Omega_{d1} D_1(t) \sigma_1^x\\
    + \Omega_{d0} U_0^{0,1}(t) \sigma_0^x + \Omega_{d1} U_1^{1,0}(t) \sigma_1^x 
    \end{split}
\end{equation}
\begin{align}
    & D_n(t) = e^{i \omega_n t} d(t) & d(t) = amp\\
    & U_n^{a,b}(t) = e^{i \omega_b t} u(t) & u(t) = amp
\end{align}
% \begin{align} 
%     \mathcal{H}/\hbar = 
%     & \sum_{i=0}^{4}\left(\frac{\omega_{q,i}}{2}(\mathbb{I}-\sigma_i^{z})
%     +\frac{\Delta_{i}}{2}(O_i^2-O_i)+\Omega_{d,i}D_i(t)\sigma_i^{X}\right) \\ 
%     & + J_{1,2}(\sigma_{1}^{+}\sigma_{2}^{-}+\sigma_{1}^{-}\sigma_{2}^{+}) 
%     + J_{3,4}(\sigma_{3}^{+}\sigma_{4}^{-}+\sigma_{3}^{-}\sigma_{4}^{+}) 
%     + J_{0,1}(\sigma_{0}^{+}\sigma_{1}^{-}+\sigma_{0}^{-}\sigma_{1}^{+}) 
%     + J_{2,3}(\sigma_{2}^{+}\sigma_{3}^{-}+\sigma_{2}^{-}\sigma_{3}^{+}) \\ 
%     & + \Omega_{d,0}(U_{0}^{(0,1)}(t))\sigma_{0}^{X} 
%     + \Omega_{d,1}(U_{1}^{(1,0)}(t)+U_{2}^{(1,2)}(t))\sigma_{1}^{X} \\ 
%     & + \Omega_{d,2}(U_{3}^{(2,1)}(t)+U_{4}^{(2,3)}(t))\sigma_{2}^{X} 
%     + \Omega_{d,3}(U_{6}^{(3,4)}(t)+U_{5}^{(3,2)}(t))\sigma_{3}^{X} \\ 
%     & + \Omega_{d,4}(U_{7}^{(4,3)}(t))\sigma_{4}^{X}
% \end{align}

OK IT SEEMS LIKE singular matrix for if we take 01 and 12 separately?
\end{document}