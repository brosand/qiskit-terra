\documentclass[12pt]{article}
\usepackage{physics}

\title{Rotating Wave Approximation for QOC}
\author{
        Ben Rosand \\
        IBM Quantum \\
        Quantum Intern -- Pulse Team\\
        Yorktown Heights, NY
}
\date{\today}


\begin{document}
\maketitle

\begin{abstract}
This is the paper's abstract \ldots
\end{abstract}

\section{Introduction}
This is time for all good men to come to the aid of their party!

\paragraph{Outline}
The remainder of this article is organized as follows.
Section~\ref{previous work} gives account of previous work.
Our new and exciting results are described in Section~\ref{results}.
Finally, Section~\ref{conclusions} gives the conclusions.

\section{Previous work}\label{previous work}
A much longer \LaTeXe{} example was written by Gil~\cite{Gil:02}.

\section{Results}\label{results}
In this section we describe the results.

\section{Rotating Wave Approximation}\label{RWA}
\subsection{Single rotation}
We follow the template laid out in [CITE FIscher] to perform the rotating 
wave approximation for an N-level system with one control field. For starters, we take the laboratory frame Hamiltonian [CITE FISCHER]
\begin{equation}\large
        H_{lab} = H_0 + \sum^M_{m=1}\Omega_m \cos(\omega_m t + \phi_m) \sum_{n'>n} g_{nn'}\sigma^x_{nn'}
\end{equation}

This generalized hamiltonian maps perfectly to our systems, with $H_0$ representing all of our time independent 
elements. $g_{nn}$ 
represents the prefactors on various drive terms (represented in our printed hamiltonians as $\Omega_n$).

Sometimes we can ignore certain transitions because they are off-resonance. I would think this 
would apply to the transitions to the 2 state for our systems, but need to DOUBLE CHECK.

The first replacement is the basic RWA, replacing the driving terms with the terms from the RWA (FISCHER 3.25).
also let $S = {(n,n')}$ which are on resonance and allowed.
\begin{equation}\label{F325}
        H_{lab} \approx H_0 + \frac{\Omega}{2} \sum_S g_{nn'} \cos(\omega t + \phi)\sigma^x_{nn'} - \sin(\omega t + \phi)\sigma^y_{nn'})
\end{equation}

Our goal here is to get rid of the oscillation $\omega t$. Again from fischer the desired transformation is:

\begin{equation}
        \ket{\psi}_{rot} = e^{-iR} \ket{\psi}_{lab}
\end{equation}

We take the derivative and obtain an equation for the transformed Hamiltonian:

\begin{equation}\label{eq:4}
        H_{rot} = e^{iR} H_{lab} e^{-iR} + \derivative{R}{t}
\end{equation}

From this equation, our goal is to find R. The core of the RWA is the following transformation:

\begin{equation}\label{eq:5}
        e^{-iR} \left \{\cos(\omega t + \phi)\sigma^x_{nn'} - \sin(\omega t + \phi)\sigma^y_{nn'}\right \}
        e^{iR} = \cos (\phi) \sigma^x_{nn'} - sin(\phi)\sigma^y_{nn'}
\end{equation}

This transformation is applied over the set $S$ in the sum in \eqref{F325}. The final step to the transformation is to find R 
such that these transformations hold. With that R in hand, we can see that \eqref{eq:4} evaluates to 
\begin{equation}\label{eq:gen_trans}
        H_{rot} = H_0 + \derivative{R}{t} + \frac{1}{2}\Omega \sum_S g_{nn'} \left \{ \cos{\phi} \sigma^x_{nn'} - sin{\phi} \sigma^y_{nn'} \right \}
\end{equation}

This is the rotating frame Hamiltonian, the key difference being the lack of oscillation on $\omega$ (carrier frequency), 
which is moved into the "generalised detuning term" $H_0 + \derivative{R}{t}$. Note that in the two-level case this term 
reduces to $\frac{1}{2} \delta \omega \sigma^z$

What does equation \eqref{eq:gen_trans} look like for IBM Q devices? All variables can be fillled in except for R,
which can be solved through another process which will be shown after.
\begin{align}
        & H_0 = H_d + H_{coupling} + H_{occupation_operator} \\
        & \Omega = \Omega_{d,i} \qquad \text{Note that this transformation is for only one drive} \\
        & g_{nn'} = 1
\end{align}
\textit{$g_{nn'}$ is technically contained in $\Omega$ because we are off resonance so only concerned with 0 to 1 transition}

\subsection{Determining transformation matrix R}
The matrix R is derived from equation \eqref{eq:5}. The two relationships that Fischer uses to make this derivation are the 
fact that R and $\sigma^z_{nn'}$ commute.

\subsection{Multiple drives and rotations}

\section{New RWA}


\section{Conclusions}\label{conclusions}
We worked hard, and achieved very little.

\bibliographystyle{abbrv}
\bibliography{main}

\end{document}