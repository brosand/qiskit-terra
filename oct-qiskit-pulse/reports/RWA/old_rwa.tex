\documentclass[12pt]{article}
\usepackage{physics}
\usepackage{bbold}

\title{Rotating Wave Approximation for QOC}
\author{
        Ben Rosand \\
        IBM Quantum \\
        Quantum Intern -- Pulse Team\\
        Yorktown Heights, NY
}
\date{\today}


\begin{document}
\maketitle

\begin{abstract}
This is the paper's abstract \ldots
\end{abstract}

\section{Introduction}
This is time for all good men to come to the aid of their party!

\paragraph{Outline}
The remainder of this article is organized as follows.
Section~\ref{previous work} gives account of previous work.
Our new and exciting results are described in Section~\ref{results}.
Finally, Section~\ref{conclusions} gives the conclusions.

\section{Previous work}\label{previous work}
A much longer \LaTeXe{} example was written by Gil~\cite{Gil:02}.

\section{Results}\label{results}
In this section we describe the results.

\section{Rotating Wave Approximation}\label{RWA}
\subsection{Single rotation}
We follow the template laid out in [CITE FIscher] to perform the rotating 
wave approximation for an N-level system with one control field. For starters, we take the laboratory frame Hamiltonian [CITE FISCHER]
\begin{equation}\large
        H_{lab} = H_0 + \sum^M_{m=1}\Omega_m \cos(\omega_m t + \phi_m) \sum_{n'>n} g_{nn'}\sigma^x_{nn'}
\end{equation}

This generalized hamiltonian maps perfectly to our systems, with $H_0$ representing all of our time independent 
elements. $g_{nn}$ 
represents the prefactors on various drive terms (represented in our printed hamiltonians as $\Omega_n$).

Sometimes we can ignore certain transitions because they are off-resonance. I would think this 
would apply to the transitions to the 2 state for our systems, but need to DOUBLE CHECK.

The first replacement is the basic RWA, replacing the driving terms with the terms from the RWA (FISCHER 3.25).
also let $S = {(n,n')}$ which are on resonance and allowed.
\begin{equation}\label{F325}
        H_{lab} \approx H_0 + \frac{\Omega}{2} \sum_S g_{nn'} \cos(\omega t + \phi)\sigma^x_{nn'} - \sin(\omega t + \phi)\sigma^y_{nn'})
\end{equation}

Our goal here is to get rid of the oscillation $\omega t$. Again from fischer the desired transformation is:

\begin{equation}
        \ket{\psi}_{rot} = e^{-iR} \ket{\psi}_{lab}
\end{equation}

We take the derivative and obtain an equation for the transformed Hamiltonian:

\begin{equation}\label{eq:4}
        H_{rot} = e^{iR} H_{lab} e^{-iR} + \derivative{R}{t}
\end{equation}

From this equation, our goal is to find R. The core of the RWA is the following transformation:

\begin{equation}\label{eq:5}
        e^{-iR} \left \{\cos(\omega t + \phi)\sigma^x_{nn'} - \sin(\omega t + \phi)\sigma^y_{nn'}\right \}
        e^{iR} = \cos (\phi) \sigma^x_{nn'} - sin(\phi)\sigma^y_{nn'}
\end{equation}

This transformation is applied over the set $S$ in the sum in \eqref{F325}. The final step to the transformation is to find R 
such that these transformations hold. With that R in hand, we can see that \eqref{eq:4} evaluates to 
\begin{equation}\label{eq:gen_trans}
        H_{rot} = H_0 + \derivative{R}{t} + \frac{1}{2}\Omega \sum_S g_{nn'} \left \{ \cos{\phi} \sigma^x_{nn'} - sin{\phi} \sigma^y_{nn'} \right \}
\end{equation}

This is the rotating frame Hamiltonian, the key difference being the lack of oscillation on $\omega$ (carrier frequency), 
which is moved into the "generalised detuning term" $H_0 + \derivative{R}{t}$. Note that in the two-level case this term 
reduces to $\frac{1}{2} \delta \omega \sigma^z$

What does equation \eqref{eq:gen_trans} look like for IBM Q devices? All variables can be fillled in except for R,
which can be solved through another process which will be shown after.
\begin{align}
        & H_0 = H_d + H_{coupling} + H_{occupation_operator} \\
        & \Omega = \Omega_{d,i} \qquad \text{Note that this transformation is for only one drive} \\
        & g_{nn'} = 1
\end{align}
\textit{$g_{nn'}$ is technically contained in $\Omega$ because we are off resonance so only concerned with 0 to 1 transition}

\subsection{Determining transformation matrix R}
The matrix R is derived from equation \eqref{eq:5}. The two relationships that Fischer uses to make this derivation are the 
fact that R and $\sigma^z_{nn'}$ commute.

\subsection{Multiple drives and rotations}

\section{Rotating Wave Approximation for athens+}
Note that for any more qubits the system looks the same, this is the minimal example.
What does the lab Hamiltonian look like?
\begin{align} \label{eq:athens_hamiltonian}
         & \mathcal{H}/\hbar =
          \sum_{i=0}^{4}\left(\frac{\omega_{q,i}}{2}(\mathbb{I}-\sigma_i^{z})+\frac{\Delta_{i}}{2}(O_i^2-O_i)+\Omega_{d,i}D_i(t)\sigma_i^{X}\right) \\
         & + J_{1,2}(\sigma_{1}^{+}\sigma_{2}^{-}+\sigma_{1}^{-}\sigma_{2}^{+}) + J_{3,4}(\sigma_{3}^{+}\sigma_{4}^{-}+\sigma_{3}^{-}\sigma_{4}^{+}) \\ 
         & + J_{0,1}(\sigma_{0}^{+}\sigma_{1}^{-}+\sigma_{0}^{-}\sigma_{1}^{+}) + J_{2,3}(\sigma_{2}^{+}\sigma_{3}^{-}+\sigma_{2}^{-}\sigma_{3}^{+}) \\ 
         & + \Omega_{d,0}(U_{0}^{(0,1)}(t))\sigma_{0}^{X} + \Omega_{d,1}(U_{1}^{(1,0)}(t)+U_{2}^{(1,2)}(t))\sigma_{1}^{X} \\ 
         & + \Omega_{d,2}(U_{3}^{(2,1)}(t)+U_{4}^{(2,3)}(t))\sigma_{2}^{X} + \Omega_{d,3}(U_{6}^{(3,4)}(t)+U_{5}^{(3,2)}(t))\sigma_{3}^{X} \\
         & + \Omega_{d,4}(U_{7}^{(4,3)}(t))\sigma_{4}^{X} \\
\end{align}

For this example, we are only concerned with the single qubit case ( we will extend to 2q later). We are only going to model at most 2q at a time.

In this case, the lab hamiltonian looks like this:

\begin{align} \label{eq:athens_singleq}
         & \mathcal{H}/\hbar = \frac{\omega_{q,0}}{2}(\mathbb{I}-\sigma_0^{z})+\frac{\Delta_{0}}{2}(O_0^2-O_0)+\Omega_{d,0}D_0(t)\sigma_0^{X}\\
         & + J_{0,1}(\sigma_{0}^{+}\sigma_{1}^{-}+\sigma_{0}^{-}\sigma_{1}^{+}) \\
         & + \Omega_{d,0}(U_{0}^{(0,1)}(t))\sigma_{0}^{X}
\end{align}

Now the last term is the control channel, which we can ignore for the 1q situation.
in addition, we can combine all the time independent parts to form $H_0$, giving us 
\begin{equation}
         \mathcal{H}/ \hbar = H_0 + \Omega_{d_0} D_0(t) \sigma_0^x
\end{equation}

Now if we look back at \eqref{eq:gen_trans} we see clear places to plug our numbers in. R
is obtained on the side.
\subsection{What about the drive term?}
The drive term that we see in the lab frame (after droppingcounter rotating
terms and replacing with RWA) is 
\begin{equation}
        H_{lab} \approx H_0 + \frac{\Omega}{2} \sum_S g_{nn'} \cos(\omega t + \phi)\sigma^x_{nn'} - \sin(\omega t + \phi)\sigma^y_{nn'})
\end{equation}

The drive term that we see from athens is:

\begin{equation}
        \Omega_{d_0} D_0(t) \sigma_0^x
\end{equation}


I'm pretty sure that this works out fine with the RWA (part 1? Fischer says use it to get to here but it still has the t terms so not sure) and we get 

\begin{equation}
        H_{rot} = H_0 + \frac{\Omega_{d_0}}{2} \left[ \cos(\omega t + \phi)\sigma^x_{01} - \sin(\omega t + \phi)\sigma^y_{01} \right]
\end{equation}

\subsection{Finding R for single q athens}
To find R, we look at equation 3.40 from Fischer: 
\begin{equation}
        [R, \sigma^x_{nn'} = \omega t(i \omega^y_{nn'})]
\end{equation}

then we get the system of equations:
\begin{align}
        & c_{01}(2 i \sigma^y_{01}) + c_{12}(- i \sigma^y_{01}) = \omega t (i \sigma^y_{01})\\
        & c_{01}(- i \sigma^y_{12}) + c_{12}(2 \sigma^y_{01}) = \omega t (i \sigma^y_{12})\\
\end{align}

when we solve this system of equations we get:
$c_{01} = c_{12} \omega t$

Thus $R = \omega t \left[\sigma^z_{01} + \sigma^z_{12}\right]$




Therefore, our hamiltonian is almost in the rotating frame:
% \begin{equation}\label{eq:post_r_ham}
% \begin{split}
%         & H_{lab} \approx H_0 + \omega \left[\sigma^z_{01} + \sigma^z_{12}\right] \\ 
%         & + \frac{\Omega_{d_0}}{2} \left[ \cos(\omega t + \phi)\sigma^x_{01} - \sin(\omega t + \phi)\sigma^y_{01} \right]
% \end{split}
% \end{equation}

The final step is to 

\subsection{Converting to format for qutip}
In order to run qutip we need to factor out the t

we start with \eqref{eq:post_r_ham} and focus on the drive term, setting
$\cos{(\omega t + \phi)}$ to $f(t)$ and $\sin{(\omega t + \phi)}$ to $f'(t)$

\begin{equation}
\begin{split}
        & H_{rot} = H_0 + \omega \left[\sigma^z_{01} + \sigma^z_{12}\right] \\ 
        & + \frac{\Omega_{d_0}}{2} \left[ f(t) + f'(t) \right]
\end{split}
\end{equation}

\subsection{The rotating frame transformation}

Starting with equation (3.26) from Fischer, we see 
\begin{equation}\label{eq:post_r_ham}
\begin{split}
        & H_{lab} \approx H_0 + \omega \left[\sigma^z_{01} + \sigma^z_{12}\right] \\ 
        & + \frac{\Omega_{d_0}}{2} \left[ \cos(\omega t + \phi)\sigma^x_{01} - \sin(\omega t + \phi)\sigma^y_{01} \right]
\end{split}
\end{equation}
we start by making the assumption that their exists some $R$ such that

\begin{equation}
        \ket{\psi}_{rot} = e^{-iR} \ket{\psi}_{lab}
\end{equation}

We take the derivative and find the hamiltonian 

\begin{equation}
        H_{rot} = e^{-iR} H_{lab} e^{ir} + \derivative{R}{t}
\end{equation}

The terms inside of the sum in the hamiltonian are orthogonal to each other.
Therefore the $\omega$ oscillation must be removed separately.

Ultimately we seek some $R$ that allows the transformation which removes the time dependence 

\begin{equation}
        \begin{split}
                & e^{-i R}\left[ \cos(\omega t + \phi)\sigma^x_{01} - \sin(\omega t + \phi)\sigma^y_{01}\right] e^{iR} \\
                & = \cos{(\phi)} \sigma^x_{01} - \sin{(\phi)} \sigma^y_{01}
        \end{split}
\end{equation}

... find the R section

Now that we have R, we get 

\begin{equation}\label{eq:rotating_frame_ham}
        \begin{split}
        & H_{rot} = H_0 + \omega \left[\sigma^z_{01} + \sigma^z_{12}\right] \\
        & + \frac{\Omega_{d_0}}{2} \qty[\cos{\phi} \sigma^x_{01} - \sin{\phi} \sigma^y_{01}]
        \end{split}
\end{equation}

In order to convert this to a format more easily digestible for qutip we need to replace the cos and sin, which we do

$H(t) = H_d + \sum(u_1(t)H_{c1} + u_2(t)H_{c2} + ....)$

Goal:
\begin{align}
        & \cos{\phi(t)} \sigma^x_{01} - \sin{\phi(t)} \sigma^y_{01} = a*f(t) +b*g(t)\\
        & \cos{\phi(t)}^2 + \sin{\phi}^2 = 1\\
        &  f(t) \neq f(s(t)) \qquad \& \qquad g(t) \neq g(u(t))\\
        & s(t), u(t) \notin \text{first order}
\end{align}

\begin{align}
        & f(t) = a \cos{x} - b \sin{x} \\
        & f(t) = \frac{a}{2} \qty[ e^{ix} + e^{-ix}] - \frac{b}{2i} \qty[e^{ix} - e^{-ix}]
        & D(t) = f(t)
\end{align}

\section{Conclusions}\label{conclusions} We worked
hard, and achieved very little.

\section{Notes that may be useful}
\begin{align}
        & H_c = [cos(\phi) * \sigma_x + sin(\phi) * \sigma_y()] \\
        & d(t)*H_c = H(t) \\
        & d(t) [cos(\phi(t)) * \sigma_x() + sin(\phi) * \sigma_y()] = H(t) \\
        & f(t) = d(t) * cos(\phi(t)) \\
        & [d(t) cos(\phi(t)) * \omega_d0 * \sigma_x(), d(t) sin(\phi(t)) * \omega_d0 * \sigma_y()] \\
        & [f(t) * \omega_d0 * \sigma_x(), f1(t) * \omega_d0 * \sigma_y()] \\
        & f(t) = d(t) * cos(\phi(t)) \\
        & f1(t) = d(t) * sin(\phi(t)) \\
        &  D(t) = f(t) + i * f1(t)
\end{align}
\bibliographystyle{abbrv}
\bibliography{main}

\end{document}